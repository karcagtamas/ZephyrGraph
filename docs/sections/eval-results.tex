\chapter{Evaluation and Results}
\label{ch:eval-results}

This chapter presents a comprehensive evaluation of the application's performance and functionality, focusing on how well it meets the design goals. We assess its capabilities based on performance metrics, usability, and accuracy of transformations, followed by a summary of test results and a discussion of how well it aligns with expectations.

\section{Evaluation Metrics}

The evaluation is based on three primary metrics that capture the core goals of the application:

\begin{itemize}
    \item \textbf{Performance Efficiency}: Measures processing time for generating decision tables and handling complex rule sets.
    \item \textbf{Transformation Accuracy}: Assesses how precisely the application converts logical expressions into decision tables and whether the results are correct and reliable.
    \item \textbf{Scalability and Responsiveness}: Examines how the application handles increased loads, including larger datasets and nested logical structures.
\end{itemize}

These metrics were used to test the application under different scenarios, ranging from simple rule sets to complex, nested conditions.

\section{Performance Efficiency}

Performance efficiency was tested using various datasets to determine the time required to transform and process rules into decision tables.

\begin{itemize}
    \item \textbf{Small Datasets}: For datasets with fewer than 50 rules, transformation times remained consistently under 1 second. This met the efficiency target for interactive performance in a development environment.
    \item \textbf{Medium Datasets}: With 50-200 rules, average processing times increased but stayed within 2-3 seconds, still within acceptable limits for efficient use.
    \item \textbf{Large Datasets}: For complex datasets (200+ rules), processing times exceeded 5 seconds in certain cases. Optimizations in parsing and transformation logic would improve responsiveness here.
\end{itemize}

Overall, the application demonstrated efficient performance, although large datasets highlighted areas for potential optimization.

\section{Transformation Accuracy}

The accuracy of logical transformations is central to the application's purpose, ensuring that complex rules are translated into correct logical representations and decision tables. To evaluate accuracy, we conducted comparison tests against expected results for different logical structures:

\begin{itemize}
    \item \textbf{Single-Cause Rules}: The application correctly mapped these to simple decision table entries, achieving 100\% accuracy.
    \item \textbf{Nested and Complex Rules}: For rules with nested logical operators (AND, OR, NOT), the application reliably transformed them into the expected Disjunctive Normal Form (DNF), preserving logical integrity in all cases.
    \item \textbf{Error Handling}: The application successfully detected and notified users of syntax and logical errors, aiding in real-time corrections during rule entry.
\end{itemize}

This evaluation confirms the application's accuracy in translating logical rules into structured formats, meeting the necessary standards for decision table generation and external test case generation.

\section{Scalability and Responsiveness}

Scalability and responsiveness were assessed by examining how the application responded to concurrent user input and increasing data volume. Key findings are:

\begin{itemize}
    \item \textbf{Concurrent Usage}: Initial tests with multiple simulated users showed minimal delay, but higher concurrency levels (50+ simultaneous sessions) led to noticeable performance drops, suggesting a need for further scalability solutions.
    \item \textbf{Large Data Volumes}: The system managed larger rule sets well in lower-concurrency environments, but increased user activity in combination with large datasets introduced latency. Future implementations of containerized environments (e.g., Kubernetes) could support scalable deployment to address this.
    \item \textbf{Load Balancing}: Testing with load balancers revealed potential improvements in maintaining responsiveness under load. With load balancing, response times improved, validating this as a viable approach for enhanced scalability.
\end{itemize}

These results suggest that while the application handles most loads effectively, there is room to expand scalability through optimized infrastructure.

\section{Summary of Results}

The evaluation demonstrates that the application achieves its primary goals of efficient rule processing, accurate logical transformation, and scalability under moderate loads. While the results are promising, especially for small to medium datasets, further enhancements are recommended for high-concurrency environments and larger data sets. The results underscore the application's strengths in accuracy and usability, while also identifying performance improvements for high-demand scenarios.

In conclusion, the application meets its functional objectives and provides a solid foundation for ongoing development, with additional scaling and performance optimizations as areas for future enhancement.